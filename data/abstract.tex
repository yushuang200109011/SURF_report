% !TeX root = ../thuthesis-example.tex

% 中英文摘要和关键字

\begin{abstract}
  四旋翼飞行器具有敏捷的动力学特性,可以在复杂的环境中高速飞行,具有广泛的研究价值和应用价值。但由于其动力学的非线性和未知环境的复杂性,其自主导航是一项具有挑战性的任务,迄今为止只有训练有素的人类飞行员才能发挥其极限性能。本研究提出了一种基于强化学习端到端的方法,使四旋翼飞行器仅用机载传感和计算设备便能自主通过未知的环境。为解决观测空间大、无人机动力学非线性和环境复杂带来的探索空间大,算法难以收敛等问题,本研究提出了先使用特权信息预训练再

  % 关键词用“英文逗号”分隔,输出时会自动处理为正确的分隔符
  \thusetup{
    keywords = {自主无人系统导航, 运动和路径规划, 强化学习},
  }
\end{abstract}

\begin{abstract*}
  An abstract of a dissertation is a summary and extraction of research work and contributions.
  Included in an abstract should be description of research topic and research objective, brief introduction to methodology and research process, and summary of conclusion and contributions of the research.
  An abstract should be characterized by independence and clarity and carry identical information with the dissertation.
  It should be such that the general idea and major contributions of the dissertation are conveyed without reading the dissertation.

  An abstract should be concise and to the point.
  It is a misunderstanding to make an abstract an outline of the dissertation and words “the first chapter”, “the second chapter” and the like should be avoided in the abstract.

  Keywords are terms used in a dissertation for indexing, reflecting core information of the dissertation.
  An abstract may contain a maximum of 5 keywords, with semi-colons used in between to separate one another.

  % Use comma as separator when inputting
  \thusetup{
    keywords* = {Autonomous vehicle navigation, Motion and path planning, Reinforcement learning},
  }
\end{abstract*}
