% !TeX root = ../thuthesis-example.tex

\chapter{无人机硬件平台设计}

本节主要介绍本研究使用的无人机具体硬件设计、选型以及三代无人机性能指标上的改进。

前两代无人机的设计主要参考了常见的250轴距穿越机设计\cite{zhou2020ego},第三代无人机由于对通信、计算需求增加,对载荷的要求也更高,因此第三代无人机由河北福莱卡航空设计了其机械结构。在此也向河北福莱卡航空表示感谢。以最新的第三代无人机为例,该无人机的硬件选型如表\ref{tab_hardware}所示。三代无人机性能比较如表\ref{tab_hardware_cmp}所示(仅列举了有变化的性能参数)。

第二代相较于第一代主要增加了定位相机模块和内嵌式的VIO算法,大大增强了无人机定位的可靠性。第三代相较于第二代更新了通信系统,增加了一块计算板卡,并为此增加了轴距,改变了整个动力系统的选型。

\begin{table}
    \centering
    \begin{tabular}{cc}
    \hline
        配件名称 & 配件型号 \\ \hline
        机架 & 自主设计双层式碳纤维机架 \\ 
        电机 & ~ \\ 
        螺旋桨 & ~ \\ 
        电子调速器 & ~ \\ 
        飞行控制器 & CUAV V5+ \\ 
        电池 & Tattu 6s 3700mAh \\ 
        深度相机 & Intel Realsense d435i \\ 
        定位相机 & Intel Realsense t265 \\ 
        计算板卡 & Nvidia Xavier NX 两块 \\ 
        机载网桥 & Vonets VM300-L \\ 
        机载交换机 & Vonets VSP500 \\ \hline
    \end{tabular}
    \caption{第三代无人机硬件选型}
    \label{tab_hardware}
\end{table}

\begin{table}[!ht]
    \centering
    \begin{tabular}{cccc}
    \hline
        参数名称 & 第一代 & 第二代 & 第三代 \\ \hline
        起飞重量(kg) & 0.98 & 1.18 & ~ \\ 
        总推力(N) & 51.23 & 51.23 & ~ \\ 
        推重比 & 5.33 & 4.43 & ~ \\ 
        定位精度(m) & 0.05 & 0.02 & 0.02 \\ 
        GPU算力(TOPs) & 21 & 21 & 42 \\ 
        通信范围(m) & \~8 & \~8 & \~15 \\ 
        通信能力 & 集中式通信,依赖基站 & 集中式通信,依赖基站 & 分布式通信 \\ \hline
    \end{tabular}
    \caption{三代无人机性能比较}
    \label{tab_hardware_cmp}
\end{table}

