% !TeX root = ../thuthesis-example.tex

\chapter{部署平台设计}
一个完备的飞行部署平台应具备性能稳定、接口完善、易用性高等优点。目前市面上尚没有一种满足本研究算法部署需求的平台。本研究从硬件选型、装配再到部署系统的开发,设计了整个部署平台。本部署平台的开发参考了相关工作的开源硬件平台\cite{zhou2020ego}和控制系统\cite{Faessler18ral}。

\section{硬件平台搭建}

截止本报告撰写,为本研究设计的自主无人机已经迭代三个版本。如图\ref{}所示。这三个版本无人机的功能改进如表\ref{}所示。最新版本自主无人机的基本参数如表\ref{}所示。
% 详细的硬件平台选型及各部件基本参数如附录\ref{}所示。

\section{部署系统}
本研究基于ROS开发了一套运行于机载计算机上的部署系统。部署系统的主要功能是收集传感器的输入、提供算法接口、为飞行器保证基本的飞行能力、以及与底层执行器(在本研究中是飞行控制器)通信。本研究开发的部署系统结构如图\ref{}所示,每个方框表示一个ROS节点。

\subsection{系统输入}
系统的输入与表\ref{tab_input}所示一致。其中VIO的算法被嵌入式地集成到定位相机模块中,定位相机会直接输出飞行器的位置、姿态和速度信息。深度图的信息直接来源于深度相机模块。

为增强定位系统的稳定性,本研究同样使用飞行控制器携带的IMU读取位姿信息。部署系统与飞行控制器通过Mavros桥(\ref{mavros_bridge}节)通信,读取IMU信息并使用增强卡尔曼滤波\cite{kalman1960contributions}\cite{kalman1960new}\cite{kalman1961new}(extended Kalman filter, EKF)融合定位相机与IMU的信息,得到更可靠的飞行器位姿信息。

\subsection{算法接口}
本部署框架于AUtopilot节点内集成了模型预测控制器\cite{Falanga2018}(Model predict control, MPC)和比例-积分-微分控制器(Proportion-integration-differentiation, PID)控制器。通过封装,两个控制器均可以接受五种不同的命令输入,并将他们转换为统一的输出。输出有两种模式可供选择。输入和输出命令的具体介绍如表\ref{tab_autopilot_input}和表\ref{tab_autopilot_output}所示。

\begin{table}[!ht]
    \centering
    \begin{tabular}{ccl}
    \hline
        \textbf{输入名称} & \textbf{输入内容} & \textbf{输入描述} \\ \hline
        pos\_command & $(x,\ y,\ z)$ & 指定空间中一点,按照直线飞行至该位置。 \\ 
        velo\_command & $(v_x,\ v_y,\ v_z,\ \omega_{z})$ & 指定线速度和$z$向角速度,按照该速度飞行。 \\ 
        \multirow{3}*{ref\_command} & 一个轨迹点(Trajectory Point): & 指定轨迹点,使飞行器的状态与轨迹点尽 \\ 
        & $(x,\ y,\ z,\ \text{roll, pitch, yaw})$ & 可能接近。通常轨迹点与当前状态差别不 \\
        & 及其$1\sim3$阶时间导数 & 大,连续发送轨迹点以实现光滑控制。 \\
        traj\_command & 一串轨迹点 & 从初始位置连续飞行全部轨迹点 \\ 
        control\_command & 控制器输出的指令格式 & 控制器不工作 \\ \hline
    \end{tabular}
    \caption{集成控制器可选输入}
    \label{tab_autopilot_input}
\end{table}

\begin{table}[!ht]
    \centering
    \begin{tabular}{ccl}
    \hline
        \textbf{输出模式} & \textbf{输出内容} & \textbf{输入描述} \\ \hline
        attitude & $(\text{thrust},\ \text{roll, pitch, yaw})$ & 飞行器的总推力和目标姿态 \\ 
        bodyrates & $(\text{thrust},\ \omega_x,\ \omega_y,\ \omega_z)$ & 即CTBR命令 \\ \hline
    \end{tabular}
    \caption{集成控制器可选输出}
    \label{tab_autopilot_output}
\end{table}

\subsection{状态切换}
\subsection{Mavros桥}
\label{mavros_bridge}
\subsection{部署系统的基本性能}

%此处应包含部署系统的基础能力测试

% 模板支持 BibTeX 和 BibLaTeX 两种方式处理参考文献。
% 下文主要介绍 BibTeX 配合 \pkg{natbib} 宏包的主要使用方法。


% \section{顺序编码制}

% 在顺序编码制下,默认的 \cs{cite} 命令同 \cs{citep} 一样,序号置于方括号中,
% 引文页码会放在括号外。
% 统一处引用的连续序号会自动用短横线连接。

% \thusetup{
%   cite-style = super,
% }
% \noindent
% \begin{tabular}{l@{\quad$\Rightarrow$\quad}l}
%   \verb|\cite{zhangkun1994}|               & \cite{zhangkun1994}               \\
%   \verb|\citet{zhangkun1994}|              & \citet{zhangkun1994}              \\
%   \verb|\citep{zhangkun1994}|              & \citep{zhangkun1994}              \\
%   \verb|\cite[42]{zhangkun1994}|           & \cite[42]{zhangkun1994}           \\
%   \verb|\cite{zhangkun1994,zhukezhen1973}| & \cite{zhangkun1994,zhukezhen1973} \\
% \end{tabular}


% 也可以取消上标格式,将数字序号作为文字的一部分。
% 建议全文统一使用相同的格式。

% \thusetup{
%   cite-style = inline,
% }
% \noindent
% \begin{tabular}{l@{\quad$\Rightarrow$\quad}l}
%   \verb|\cite{zhangkun1994}|               & \cite{zhangkun1994}               \\
%   \verb|\citet{zhangkun1994}|              & \citet{zhangkun1994}              \\
%   \verb|\citep{zhangkun1994}|              & \citep{zhangkun1994}              \\
%   \verb|\cite[42]{zhangkun1994}|           & \cite[42]{zhangkun1994}           \\
%   \verb|\cite{zhangkun1994,zhukezhen1973}| & \cite{zhangkun1994,zhukezhen1973} \\
% \end{tabular}



% \section{著者-出版年制}

% 著者-出版年制下的 \cs{cite} 跟 \cs{citet} 一样。

% \thusetup{
%   cite-style = author-year,
% }
% \noindent
% \begin{tabular}{@{}l@{$\Rightarrow$}l@{}}
%   \verb|\cite{zhangkun1994}|                & \cite{zhangkun1994}                \\
%   \verb|\citet{zhangkun1994}|               & \citet{zhangkun1994}               \\
%   \verb|\citep{zhangkun1994}|               & \citep{zhangkun1994}               \\
%   \verb|\cite[42]{zhangkun1994}|            & \cite[42]{zhangkun1994}            \\
%   \verb|\citep{zhangkun1994,zhukezhen1973}| & \citep{zhangkun1994,zhukezhen1973} \\
% \end{tabular}

% \vskip 2ex
% \thusetup{
%   cite-style = super,
% }
% 注意,引文参考文献的每条都要在正文中标注
% \cite{zhangkun1994,zhukezhen1973,dupont1974bone,zhengkaiqing1987,%
%   jiangxizhou1980,jianduju1994,merkt1995rotational,mellinger1996laser,%
%   bixon1996dynamics,mahui1995,carlson1981two,taylor1983scanning,%
%   taylor1981study,shimizu1983laser,atkinson1982experimental,%
%   kusch1975perturbations,guangxi1993,huosini1989guwu,wangfuzhi1865songlun,%
%   zhaoyaodong1998xinshidai,biaozhunhua2002tushu,chubanzhuanye2004,%
%   who1970factors,peebles2001probability,baishunong1998zhiwu,%
%   weinstein1974pathogenic,hanjiren1985lun,dizhi1936dizhi,%
%   tushuguan1957tushuguanxue,aaas1883science,fugang2000fengsha,%
%   xiaoyu2001chubanye,oclc2000about,scitor2000project%
% }。
