% !TeX root = ../thuthesis-example.tex

\chapter{总结与展望}
\label{final}

在本研究中,我们从设计一种高性能自主导航算法的角度出发。提出了一种基于强化学习的自主导航算法框架,为了提升算法性能和收敛速度,本研究提出了三段式训练方法以及固定编码器和域随机化两个训炼技巧。此训练方法大大提高了算法的训练效率。为了更好的完成算法训练和最终的部署展示,本研究还配套开发了高并行度的仿真器和性能稳定的部署平台。仿真平台和部署系统不仅为算法提供了高性能可靠的保障,还将作为本课题组重要基础平台,为后续研究提供帮助。

\section{研究结论}

无人机在未知环境下的自主导航任务是极具挑战的系统性任务。一方面无人机需要运行高性能的算法以在模糊的观测中提取有效信息并规划正确的路径,另一方面无人机需要稳定可靠的机械和控制系统准确执行算法给出的指令并保障高飞行品质。无人系统的任何一个环节出现问题都将导致飞行性能严重下降、任务难以完成。

强化学习作为一类具有强拟合能力的人工智能方法,其处理高动态非线性的能力能够满足无人机自主导航任务的需求。然而强化学习需要及其大量的交互式数据作为训练基础。高并行度的仿真器几乎是强化学习的标配。本研究利用并行化的思想,巧妙改进了现有的仿真器并提供了局域网内分布式训练的能力。这一改进将数据采集效率提升了一个数量级,这是本项目能够完成的重要保障。

为了保障算法在飞行平台上能够快速计算,本研究设计了一种端到端的主动导航算法框架,本框架在执行时间上位于同类算法的前列。端到端的主动导航算法涵盖了部分感知、全部决策和部分控制的任务,模型结构复杂、参数量较大,直接训练所需的数据和时间成本都会增加。三段式训练方法将框架中参数量最大的编码器部分提取出来用点云先做单独的预训练,然后单独做点云输入的强化学习,训练规划器部分,再使用编码器对齐的方式完成深度图编码器的训练。在三个步骤中引入了提高训练效率的点云编码器部分,并通过不同手段使框架每一部分都得到了训练。这一方法将算法收敛速度提升了一个数量级。

此外,本项目结合强化学习的特点和自主导航任务的特点,提出了固定编码器和域随机化两个训练技巧。避免了训练前期无规律探索可能带来的性能损失。有效提升了算法的收敛性能和泛化性,在速度的泛化性上本方法处于同类算法前列。

为了更好呈现本算法在实机部署中的性能,本研究参照穿越机、其它自主导航无人机设计了三代无人机硬件平台,搭配PX4开源飞行控制器,提供高推重比灵活的飞行性能。本研究还在无人机上运行了一套部署平台,该平台提供稳定的飞行能力和控制能力。平台集成的控制器性能和稳定性在同类型无人机中也处于前列。我们将本算法搬上了实机平台并进行了几次测试,算法能够完成基本的导航任务。

\section{未来工作展望}

算法层面,本研究涉及的自主导航算法仅对速度这一个指标的泛化性提出了优化,但有相关研究提出利用强化学习域随机化技术可以提高更多参数的泛化性,例如不同型号的无人机、不同飞行场景和不同飞行任务。本研究可以在未来工作中尝试更复杂的泛化性研究。

算法实机部署实验中,由于时间限制我们还没有进行广泛的高性能的测试,仅测试了室内场景低速飞行的情况,未来应在不同的、更大的场地内完成高速飞行的实验以测试算法的性能上限。此外,我们尚没有对实验进行定量的分析,还没有和同类先进算法比较真实的飞行性能。这也是本工作未来必须要完成的测试。

就整个系统而言,本自主导航算法虽然完成了基本的自处导航任务,但对于一些较为特殊但又时常出现的场景还没有考虑到。从飞行轨迹图中可以看出,飞行器在加速、减速、过急转弯等速度变化较大、观测变化也较大的场景中仍然表现不佳。但这些场景却是实际飞行中难以避免的场景。在未来的工作中也需要对此类问题进行针对性优化。