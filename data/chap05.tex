% !TeX root = ../thuthesis-example.tex

\chapter{实验结果与结论}
\label{result}

前述几章讨论了本研究在仿真器上的改进、在自主导航算法和训练方法上的设计以及开发的部署平台。本研究的主要创新点是提出了一种基于强化学习的自主导航算法,以及提升其效率的三段式训练方法,其它两部分内容都是为了配合该自主导航算法,使整个系统发挥出更好的性能。本章将首先展示自主导航算法和三段式训练方法在训练和仿真中展现出的优良性能,然后展示其在真机部署实验中的效果。

由于本研究是系统性的研究,因此最终性能尤其是实机部署中的性能与整个系统构建有关,例如使用何种传感器、飞行控制器等。因此在介绍有关结论时将尽可能详细介绍实验的环境以及实验的设置,并尽可能地测试每个组件单独的性能以给出参考。

\section{三段式训练方法}

三段式训练方法主要解决了原有使用深度图输入的强化学习速度慢的问题,其主要性能体现在训练速度上。本节将主要展示三段式训练方法第一、三步骤在训练过程中起到的效果,并对比三段式训练方法与使用深度图作为输入直接强化学习的训练方法在训练效率上的优势。

\subsection{预训练点云编码器}

这一训练阶段我们使用了自监督学习的方法训练了点云编码器。

\subsection{编码器对齐}

本阶段使用的
%一张超参数表

\subsection{三段式学习方法对效率的提升}

%一张训练时间表格

\section{自主导航算法性能}

% 超参数表格

\subsection{训练效果可视化}

% at 2 speed metric and fake traj

\subsection{消融实验}

% at 2 speed 消融实验

\section{部署平台基本飞行性能}

% 8字飞行,PID和MPC性能差距 ROSBAG

\section{真机部署}

% 室内穿越图片

